\documentclass[12pt]{scrreprt}

%Seitenabstände anpassen: oben-unten-links-rechts: 2,5cm, 2,5cm, 3cm, 2cm.
\usepackage{geometry}
\geometry{a4paper,tmargin=25mm,bmargin=25mm,lmargin=25mm,rmargin=25mm}

%Zeilenabstand anpassen: 1,5
\usepackage[onehalfspacing]{setspace}

\usepackage[utf8x]{inputenc}
\usepackage{parskip}
\usepackage{hyperref}
\usepackage{fancyhdr}

% Eigene Farben
\usepackage{xcolor}
\definecolor{inlineTagLink}{rgb}{0, 0.4, 0}
\definecolor{tagLink}{rgb}{0, 0, 0.8}

% Kopf- und Fußzeilen setzen
\pagestyle{fancy}
\fancyhf{}
\setlength{\headheight}{15pt}
\fancyhead[R]{\textit{\nouppercase{\rightmark}}}
\fancyfoot[C]{\thepage}

% Dokumenttitel
\title{SimpleExample}
\subtitle{package: examples}
\date{Dokument erstellt am: \today}


% Die Überschrift des Inhaltsverzeichnisses ändern
\renewcommand{\contentsname}{Inhaltsverzeichnis}

% Start des Dokumentes
\begin{document}
\pagenumbering{gobble}

\maketitle
\newpage

\pagenumbering{Roman}

\addcontentsline{toc}{chapter}{Inhaltsverzeichnis}
\tableofcontents
\newpage

% Das hier ist für den Wechsel zwischen römischen und arabischen
% Zahlen bei der Seitenzählung
\setcounter{page}{0}
\pagenumbering{arabic}


%Start des Inhaltes
\chapter{Class SimpleExample}
\label{examples.SimpleExample}




\textbf{public class SimpleExample implements Runnable}


This is a very basic description of an outer Java -class. It contains absolutely no special characters, except for these :\{\}()[]©²³\textbackslash[T ]/. Those special characters are escaped and don 't result in a LaTex conflict.


\textbf{@author:}

\quad\quad Lutz Winkelmann, Bj örn B öing

\textbf{@see:}

\quad\quad \hyperref[examples.SimpleExample.My22Class:getx]{\color{tagLink}My awesome link}

\textbf{@version:}

\quad\quad 1.0- nightlybuild


\section{Method myMethod}
\label{examples.SimpleExample:myMethod}




\textbf{public void myMethod(int a)}


This is some description text with an inlinetag referring to the Object -class \hyperref[Object]{\textbf{\color{inlineTagLink}}}.



\textbf{@param a:}

\quad\quad A simple parameter



\section{Field allNames}
\label{examples.SimpleExample:allNames}




\textbf{private ArrayList\textless String\textgreater  allNames}


This ArrayList holds all the current names.



\section{Field favoriteName}
\label{examples.SimpleExample:favoriteName}




\textbf{private String favoriteName}


This string stores the favorite name.



\section{Field counter}
\label{examples.SimpleExample:counter}




\textbf{private static int counter}






\section{Constructor SimpleExample}
\label{examples.SimpleExample:SimpleExample}




\textbf{SimpleExample(String name)}








\section{Method run}
\label{examples.SimpleExample:run}




\textbf{public void run()}


This method is here to have an "implements "in the class signature and doesn 't do anything else.





\section{Method getAllNames}
\label{examples.SimpleExample:getAllNames}




\textbf{protected ArrayList\textless String\textgreater  getAllNames()}








\section{Method addNames}
\label{examples.SimpleExample:addNames}




\textbf{public void addNames(String... name)}


This method can be used to add 1 to n names. It shows the support of the varargs operator.



\textbf{@param name:}

\quad\quad A list of names



\section{Method setFavoriteName}
\label{examples.SimpleExample:setFavoriteName}




\textbf{public void setFavoriteName(String name)}


This method can be used to set your favorite name



\textbf{@param name:}

\quad\quad Your new favorite



\section{Method main}
\label{examples.SimpleExample:main}




\textbf{public static void main(String[] args)}


Here is our main method which doesn 't make much sense, but includes a parameter.



\textbf{@param args:}

\quad\quad An array of arguments



\section{Class My22Class}
\label{examples.SimpleExample.My22Class}

@Description(key ="description", value ="Some fancy text to show this strange description annotation which is"+"added to Javadoc")



\textbf{public static abstract class My22Class}


We can even use and link inline tags \hyperref[examples.AnotherClass]{\textbf{\color{inlineTagLink} and this is the link}}. As you can see, this links to an external class, but it has to be in the same document.



\subsection{Method getx}
\label{examples.SimpleExample.My22Class:getx}




\textbf{abstract int getx(String a)}








\subsection{Constructor My22Class}
\label{examples.SimpleExample.My22Class:My22Class}




\textbf{My22Class() throws IOException}


This is a basic constructor which throws an IOException




\textbf{@throws IOException:}

\quad\quad some serious exception


\subsection{Method readIO}
\label{examples.SimpleExample.My22Class:readIO}




\textbf{private String readIO() throws IOException}


This method simulates some IO -Actions to show a throwing tag.


\textbf{@serialData:}

\quad\quad Much serialized, such WOW! 
         * @returnAbeautifulstring



\textbf{@throws IOException:}

\quad\quad Watch out!


\subsection{Method coolMethod}
\label{examples.SimpleExample.My22Class:coolMethod}

@com.sun.org.glassfish.gmbal.Description("Here my value"+"over multiple lines to show an even stranger annotation")



\textbf{public void coolMethod(String a, int x, int y, HashMap\textless Integer,String\textgreater  map)}


This method is to show a long list of params in combination with an annotation following the Javadoc



\textbf{@param x:}

\quad\quad The best value

\textbf{@param y:}

\quad\quad The worth value

\textbf{@param a:}

\quad\quad A cool string

\textbf{@param map:}

\quad\quad A map with really important information



\section{Method myMethod}
\label{examples.SimpleExample:myMethod}




\textbf{public static void myMethod(ArrayList\textless HashMap\textless My22Class,Integer[]\textgreater \textgreater  a, int b, String notDocumented)}


This method includes some more "logic "to show that everything inside of a method is skipped. Moreover there is the parameter "notDocumented "which isn 't listed in the Javadoc params to force a warning message.



\textbf{@param a:}

\quad\quad A really huge and abstract object

\textbf{@param b:}

\quad\quad Some useless number

\textbf{@param c:}

\quad\quad This is not allowed



\section{Method anotherMethod}
\label{examples.SimpleExample:anotherMethod}




\textbf{public ArrayList\textless String\textgreater  anotherMethod(String[] a, int b)}








\chapter{Class AnotherClass}
\label{examples.AnotherClass}




\textbf{class AnotherClass}


This class is a basic example to show some Javadoc.


\textbf{@author:}

\quad\quad Wuppi Fluppi




\end{document}