\documentclass[12pt]{scrreprt}

%Seitenabstände anpassen: oben-unten-links-rechts: 2,5cm, 2,5cm, 3cm, 2cm.
\usepackage{geometry}
\geometry{a4paper,tmargin=25mm,bmargin=25mm,lmargin=25mm,rmargin=25mm}

%Zeilenabstand anpassen: 1,5
\usepackage[onehalfspacing]{setspace}

\usepackage[utf8x]{inputenc}
\usepackage{parskip}
\usepackage{hyperref}
\usepackage{fancyhdr}

% Kopf- und Fußzeilen setzen
\pagestyle{fancy}
\fancyhf{}
\setlength{\headheight}{15pt}
\fancyhead[R]{\textit{\nouppercase{\rightmark}}}
\fancyfoot[C]{\thepage}

% Die Farbe von Zitatverlinkungen auf "Blau" setzen
\hypersetup{
 colorlinks=true,
 citecolor=blue,
 linkcolor=black,
 urlcolor=blue
}

% Dokumenttitel
\title{SimpleExample}
\subtitle{package: examples}
\date{Dokument erstellt am: \today}


% Die Überschrift des Inhaltsverzeichnisses ändern
\renewcommand{\contentsname}{Inhaltsverzeichnis}

% Start des Dokumentes
\begin{document}
\pagenumbering{gobble}

\maketitle
\newpage

\pagenumbering{Roman}

\addcontentsline{toc}{chapter}{Inhaltsverzeichnis}
\tableofcontents
\newpage

% Das hier ist für den Wechsel zwischen römischen und arabischen
% Zahlen bei der Seitenzählung
\setcounter{page}{0}
\pagenumbering{arabic}


%Start des Inhaltes
\chapter{class SimpleExample}
\label{examples.SimpleExample}




public class SimpleExample extends AnotherClass implements Runnable


This is a very basic description of an outer Java- class. It contains absolutely no special characters, except for these:\{\}()[]©²³\textbackslash[ T]/. Those special characters are escaped and don' t result in a LaTex conflict.


\textbf{@see:}

\quad\quad \hyperref[examples.SimpleExample.My22Class:getx]{text}


\section{Methode run}
\label{examples.SimpleExample:run}




public void run()


This method is here to have an" implements" in the class signature and doesn' t do anything else.





\section{Methode getAllNames}
\label{examples.SimpleExample:getAllNames}




protected ArrayList\textless String\textgreater  getAllNames()








\section{Methode setFavoriteName}
\label{examples.SimpleExample:setFavoriteName}




public void setFavoriteName(String name)


This method can be used to set your favorite name



\textbf{@param name:}

\quad\quad Your new favorite



\section{Methode main}
\label{examples.SimpleExample:main}




public static void main(String[] args)


Here is our main method which doesn' t make much sense, but includes a parameter.



\textbf{@param args:}

\quad\quad An array of arguments



\section{class My22Class}
\label{examples.SimpleExample.My22Class}

@Description(key="description", value="Some fancy text to show this strange description annotation which is"+"added to Javadoc")



public static abstract class My22Class






\subsection{Methode getx}
\label{examples.SimpleExample.My22Class:getx}




 abstract int getx(String a)








\subsection{Methode readIO}
\label{examples.SimpleExample.My22Class:readIO}




private String readIO() throws IOException


This method simulates some IO- Actions to show a throwing tag.




\textbf{@throws IOException:}

\quad\quad Watch out!


\subsection{Methode coolMethod}
\label{examples.SimpleExample.My22Class:coolMethod}

@com.sun.org.glassfish.gmbal.Description("Here my value"+"over multiple lines to show an even stranger annotation")



public void coolMethod(String a, int x, int y, HashMap\textless Integer,String\textgreater  map)


This method is to show a long list of params in combination with an annotation following the Javadoc



\textbf{@param x:}

\quad\quad The best value

\textbf{@param y:}

\quad\quad The worth value

\textbf{@param a:}

\quad\quad A cool string

\textbf{@param map:}

\quad\quad A map with really important information



\section{Methode myMethod}
\label{examples.SimpleExample:myMethod}




public static void myMethod(ArrayList\textless HashMap\textless My22Class,Integer[]\textgreater \textgreater  a, int b, String notDocumented)


This method has some more" logic" included to show that everything inside of a method is skipped. Moreover there is the parameter" notDocumented" which isn' t listed in the Javadoc params.



\textbf{@param a:}

\quad\quad A really huge and abstract object

\textbf{@param b:}

\quad\quad Some useless number



\section{Methode anotherMethod}
\label{examples.SimpleExample:anotherMethod}




public ArrayList\textless String\textgreater  anotherMethod(String[] a, int b)








\chapter{class AnotherClass}
\label{examples.AnotherClass}




 class AnotherClass








\end{document}